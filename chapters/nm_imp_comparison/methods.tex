\section{Methods}

\subsection{Control Optimization}
To obtain suitable parameters for the neuromuscular and impedance control
methods we rely on the dueling bandits optimization approach outlined in
\cref{sec:bandit_optimization}. We make a few modifications to the approach
outlined previously. First, whereas in \cref{sec:bandit_optimization} we
optimized control parameters to match gait data at different speeds to achieve a
speed-adaptive control, in this work, we optimize control parameters to match
both undisturbed and disturbed gait in order to obtain robust control
parameters. We use the dataset provided by \citet{moore2015elaborate}, which
provides gait data for undisturbed walking and walking with treadmill velocity
disturbances. 

For the neuromuscular control, we use the black-box CMA-ES
optimizer~\citep{hansen2006cma} to obtain parameters that can reproduce the
behavior of each subject in the gait datset. We optimize the parameters listed
in \cref{tab:nm_params_treadmill_exp} to minimize the following cost function:
\begin{margintable}    
    \centering
    \normalsize
    \begin{tabular}{ll}
        \multicolumn{2}{l}{Optimized Parameters} \\
        \midrule
        $F_\tn{max}^\tn{ham}$              & $^{F+}G_\tn{ham}^\tn{ham}$   \\
        $F_\tn{max}^\tn{vas}$              & $^{F+}G_\tn{vas}^\tn{vas}$   \\
        $F_\tn{max}^\tn{gas}$              & $^{F+}G_\tn{gas}^\tn{gas}$   \\
        $F_\tn{max}^\tn{sol}$              & $^{F+}G_\tn{sol}^\tn{sol}$   \\
        $F_\tn{max}^\tn{ta}$               & $^{F-}G_\tn{sol}^\tn{ta}$    \\
        $^\tn{off}l_\tn{ta}^\tn{ta}$       & $^{L+}G_\tn{ta}^\tn{ta}$     \\
        $^\tn{off}\phi_\tn{knee}^\tn{vas}$ & ${^\phi}G_\tn{knee}^\tn{vas}$ \\
        $S_0^\tn{vas}$                     & $\epsilon_\tn{SE}^\tn{ap}$   \\
        $S_0^\tn{ham}$                     & $F_\tn{init}^\tn{ham}$       \\
    \end{tabular}
    \caption[Parameters optimized for parameter set generation for experiment
    comparing neuromuscular and impedance control]{Optimized parameters,
    $\Gamma$. We optimize 18 parameters. $F_\tn{max}^m$ refers to muscle $m$'s
    maximum isometric force, $S_0^m$ is muscle $m$'s pre-stimulation,
    $^{\tn{signal}G_n^m}$ is the gain on a feedback signal from muscle $n$
    acting on muscle $m$, $\epsilon_\tn{SE}^\tn{ap}$ is the tendon reference
    strain of the ankle plantarflexors (sol and gas) and $F_\tn{init}^\tn{ham}$
    is the initial force in the hamstring MTU at
    heelstrike.\label{tab:nm_params_treadmill_exp}
\end{margintable}

\subsection{Disturbance}

\subsection{Experimental Protocol}
