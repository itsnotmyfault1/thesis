\section{Methods}

\subsection{Control Optimization}
To obtain suitable parameters for the neuromuscular and impedance control
methods we rely on the dueling bandits optimization approach outlined in
\cref{sec:bandit_optimization}. We make a few modifications to the approach
outlined previously. First, whereas in \cref{sec:bandit_optimization} we
optimized control parameters to match gait data at different speeds to achieve a
speed-adaptive control, in this work, we optimize control parameters to match
both undisturbed and disturbed gait in order to improve the robust a control as
possible. We utilize the gait data provided by \citet{moore2015elaborate}, which
provides gait data during undisturbed walking and walking with treadmill
velocity disturbances. 

Second, we slightly simplify the neuromuscular prosthesis control that was used
in \cref{sec:preference_optimization} by removing the BFsH muscle and returning
to the set of reflexes listed in

optimize the parameters
listed in \cref{tab:nm_params_treadmill_exp} to minimize the follow cost

\begin{margintable}    
    \centering
    \begin{tabular}{ll}
        \hline
        HAM $F_\tn{max}$                    & HAM F+ Gain         \\
        VAS $F_\tn{max}$                    & VAS F+ Gain         \\
        GAS $F_\tn{max}$                    & GAS F+ Gain         \\
        SOL $F_\tn{max}$                    & SOL F+ Gain         \\
        TA  $F_\tn{max}$                    & SOL on TA F- Gain   \\
                                            & TA L+ Gain          \\
        TA $l_\tn{ce}^\tn{offst}$           & VAS $S_0$           \\
                                            & HAM $S_0$           \\
    \end{tabular}
    \caption{Optimized parameters, $\Gamma$. Speed-independent parameters use a
    single value for all speeds, while speed dependent parameters have distinct
    values for \unitfrac[0.8, 1.2, and 1.6]{m}{s} gaits. Consequently, in total
    we optimize 43 parameters. $F_{max}$ refers to a muscle's maximum isometric
    force, $\phi_0$ is a parameter used for muscle moment arm calculations, and
    $S_0$ is a muscle's pre-stimulation.}\label{tab:nm_params_treadmill_exp}
\end{margintable}

\subsection{Disturbance}

\subsection{Experimental Protocol}
