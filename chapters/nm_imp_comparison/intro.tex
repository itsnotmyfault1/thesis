\section{Introduction}

To date, there have been many proposed controllers for prostheses, which we
reviewed in \cref{sec:back_walking_control}. However, there has been a dearth of
studies directly comparing the merits and detriments of these different
strategies. 

In this work, we provide one such study by comparing the neuromuscular (NM) and
impedance (IMP) control strategies in a similar manner as in
\cref{sec:control_sim}.  We seek to make this comparison as objective as
possible. To do this, we minimize the experimenter's influence on controller
parameter selection by using the dueling bandits parameter selection method
presented in \cref{sec:preference_optimization}. This method comprises of two
parts: 1) We first generate parameters for both controllers through offline
optimizations that try to match the controller output to able-bodied gait data
of different subjects. 2) We use a preference-based optimization that allows
users to select their preferred parameter set. Finally, we also replace the hand
tuning of offset angles that we used in \cref{sec:preference_optimization}
(\cref{eq:bias_param}) with an iterative learning procedure.

We are primarily interested in potential robustness improvements provided by
neuromuscular prosthesis control, as predicted by the simulation results
presented in \cref{sec:control_sim}. To examine controller robustness, we tested
both controllers with able-bodied subjects walking with their preferred
parameters at constant speed and with treadmill velocity disturbances. We then
evaluate the user ratings, number of falls, reasons for falls, and gait
variability for each condition. 

We also hoped to compare neuromuscular and impedance control to the continuous
phase-based control proposed by \citet{quintero2016preliminary}. However, we
were unable to achieve a consistent walking pattern with this control strategy.
We present our results trying to implement this control strategy in
\cref{sec:nm_vs_imp_phase_imp}.
