\section{Discussion}

In this work, we sought to objectively compare the robustness, user preferences,
and reasons for falls of the neuromuscular and impedance control strategies for
powered, robotic knee and ankle prostheses. Overall, we found that users rated
the neuromuscular control more highly than impedance control and using impedance
control led to significantly more falls compared to walking without a
prosthesis. While we did observe more falls with the impedance control across
all subjects than with the neuromuscular controller, when considering the
question ``did individual users fall more often with impedance control than with
neuromuscular control?'', these differences were not significant. The only
measure of gait stability in which a significant difference between the
neuromuscular and impedance controllers was measured was the torso pitch
variability, which neuromuscular control significantly reduced in the
disturbance case compared to impedance control.

Categorizing the falls by their type gives more insight into differences between
the controllers. There were reasons for falls with each controller that did not
exist for the other. For NM control, missed transitions between stance and swing
caused three falls. While these falls could be directly attributed to the
leg-angle threshold in the high level state machine that governs the
stance/swing state transition (\cref{fig:stance_swing_state_machine}), that
these falls only occurred with neuromuscular control suggests a causal
difference in the two strategies. One possible reason that these falls did not
occur with impedance control is that the discrete transition between the second
and third stance phases generates a sudden increase in the ankle plantarflexion
torque. When walking with impedance control, subjects may have waited to feel
this transition before beginning swing. In contrast, neuromuscular control gives
no such obvious transition and thus users may attempt to enter swing too early.
However, with better sensing of ground reaction forces, the threshold on leg
angle would not be necessary, and thus this problem would be resolved.

While impedance control's discrete phase transitions may have helped users avoid
missing the stance to swing transition, it directly caused two other failure
modes. The first, missed transitions between the second and third phases of
stance, occurred if the user did not dorsiflex the ankle enough to trigger the
transition. This failure could lead to trips during swing or a later loss of
balance. The second, the knee collapse failure mode, happened if the impedance
controller switched to the third phase of stance too early, which could cause a
sudden reduction in knee extension torque. The fact that individual subjects
fell for both of these reasons suggests that we cannot fix these failure modes
by simply tuning the ankle angle threshold that governs the phase transition.
Decreasing the threshold to prevent missed transitions would likely cause more
knee collapses. Conversely, increasing the threshold would likely cause more
missed transitions. 

Finally, users suffered from trips during swing when using both stance control
strategies, which were both paired with the same minimum-jerk trajectory
generation swing control strategy. However, these trips occurred three times
more frequently with impedance stance control than with neuromuscular stance
control. Many of the trips that occurred with impedance control were preceded by
a missed transition between the second and third stance phases. Neuromuscular
control, in contrast, is smooth throughout stance with no discrete transitions,
and thus may transition to swing more consistently and cause fewer swing trips.
Nevertheless, even with its smoother stance phase, subjects still tripped during
swing several times with the Neuromuscular control. Therefore, in
\crefrange{sec:swing_control_planning_intro}{sec:swing_control_planning_discuss}
we seek to explicitly minimize the risk of tripping by using estimates of the
current and future hip height and orientation to plan knee and ankle swing
trajectories that avoid premature ground contact.

The smooth stance phase of the neuromuscular model, which eliminates failure
modes such as knee collapses and missed stance phase transitions and may help
reduce swing trips, comes at the cost of dramatically increased model
complexity. The implemented impedance controller has 20 parameters: 3 stance
phases $\times$ 2 joints $\times$ 3 parameters per joint per phase $+$ 2
transition parameters. In contrast, the implemented neuromuscular control is
more than 4 times as complex as it has 80 parameters: 54 defining
muscle-specific mechanical properties, 9 defining shared muscle properties, and
17 defining the neural reflexes. 

Of these 80, we chose to only optimize 18 when generating parameter the sets in
order to avoid local minima and to complete the optimization in a reasonable
amount of time. The choice of which 18 parameters to choose was based on trial
and error and prior experience with the model. In the clinical setting, the lack
of transparency about the function of and interdependencies between the 80
parameters may make practical application of neuromuscular control difficult.
Therefore, in order to achieve the potential benefits of a smooth controller
that does not have discrete stance phases, while avoiding excessive complexity,
in \cref{sec:phase_estimation}, we explore an alternative approach to stance
control. This approach relies on a continuous estimate of phase and easily
interpretable models for the output behavior as a function this phase estimate.

A surprising result of this experiment is the lack of substantial differences
between suboptimal and optimal controllers. Only the in the case of impedance
control under disturbances did subjects unanimously restate their preference for
the control parameter sets they had preferred on the optimization day. The lack
of clear differences between the neuromuscular parameter sets could be the
result of the neuromuscular model generally being less sensitive to its
parameters than the impedance controller. For example, large differences in
behavior between impedance control parameter sets can result if one set of
parameters causes many missed phase transitions and another parameter set does
not. Another reason for the lack of clear difference in user ratings could be
the difference in the query. During the optimization procedure, subjects were
asked to directly prefer one parameter set to other after short
$\unit[\sim10]{sec}$ bouts of walking with each parameter set. In contrast, on
the data collection day, parameter sets were independently rated on a 1-10 scale
after 2 minutes of walking. In future work, we should check for consistency of
the preferred parameters by performing the dueling bandits optimization
procedure on multiple days in order to see if the users preferences are
consistent from day to day.

Our simulated results presented in \cref{sec:control_sim} predicted a larger
difference between controllers that was not borne out by this experiment.  In
future work, the motion capture data collected during these trials should be
used to improve the neuromuscular model so that researchers can perform
experiments investigating prosthetic device performance with a higher likelihood
that those results translate to the real world. Reliable predictive models would
vastly reduce the time it takes to iterate prosthesis controller designs.

This study has several limitations that should be addressed in future work.
First, we only performed the study with ten subjects, which made establishing
the statistical significance of the results difficult. Limiting the number of
subjects was necessary due to the substantial practice time required for
subjects to achieve reliable handrail-free walking with the prosthesis.
Furthermore, there were several \todo{get number} subjects who could not proceed
past the first day of the experimental procedure as they could not manage to
walk without the handrails or experienced excessive discomfort in the prosthesis
adaptor. While researcher's should eventually perform adequately powered studies
comparing prosthesis controllers in the future, at this point in the development
of these controllers, there are still significant issues in the prosthesis
control that can be discovered and addressed with small $n$ studies. In the case
of a safety critical device like a prosthesis, all causes of falls should be
systematically addressed even if they are rare and their frequency cannot be
established to a high degree of confidence.

Second, we tested very specific implementations of the impedance and
neuromuscular controllers. Specifically, we tested the impedance control from
\citet{lawson2014robotic}, which uses linear impedance functions and angle
thresholds to transition between phases. However, other variations on the
impedance control have been suggested as well, including those that use
nonlinear impedance functions \citep{sup2007design,shultz2016variable}, and
those using ground reaction signals for the transitions between phases
\citep{sup2009preliminary}. The neuromuscular control structure we used was
described in \cref{sec:nm_control_prosthesis} and based on
\citet{song2015neural}. This structure encodes a specific set of reflexes.
However, there are many other feedback pathways that could be incorporated into
this model as well that may improve performance. Additional experiments will be
needed to compare amongst variations in these control strategies.

Finally, in this experiment, users needed to pick parameters out of a
pre-generated discrete set of possibilities. It is possible and likely that there
exists parameters for each control that are more robust, comfortable, and
efficient. However, finding these parameters by solving a high dimensional
optimization problem using real prosthesis feedback remains an unsolved problem
for three reasons: First, optimizations suffer from the curse of dimensionality.
Recently, \citet{wen2019online} optimized 12 parameters of a robotic
knee/passive ankle prosthesis, a significant improvement from prior approaches.
However, this approach is still insufficient for impedance control of both the
knee and ankle joints, which requires 20 parameters. In our work, the parameter
generation approach tackles the dimensionality through a large amount of offline
computation.

Second, it is unclear what objective function to use in the optimization. For
example, \citet{wen2019online} target a desired knee angle trajectory. However,
perfect tracking of this trajectory may come at the expense of robustness or
comfort. In this experiment, we allowed users to specify their own objective by
using their preferences. In future work, similar preference based approaches could
be used to identify a user's objective function based on multiple gait metrics.
An optimization approach, such as \citet{wen2019online}, could then be used to
target this objective function if it can be scaled to the required number of
dimensions.

The final challenge for performing the full optimization is that many important
features we would use to quantify gait are difficult to measure. For example, to
measure gait robustness using treadmill velocity disturbances, we required many
minutes of walking under disturbances just to observe a few falls. Metabolic
energy consumption typically takes several minutes to observe, even when using a
predictive model of its steady state value \citep{zhang2017human}. Other gait
features such as comfort are not directly quantifiable. Identification of
quickly measurable stand-ins for these metrics should be a focus of future
research.
