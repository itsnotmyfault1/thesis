\section{Bandit Approach Discussion} 

We present a new approach for online optimization of lower limb assistive
devices that uses preference feedback to incorporate the user's subjective
assessment of device behavior. The method tackles high dimensional tuning
problems by incorporating domain knowledge via an offline optimization step that
utilizes kinetic and kinematic data from intact subjects to obtain a library of
policy parameters. We find that the five subjects who completed our experiment
preferred four different parameter sets from this library. The resulting gait
patterns resembled intact human gait data and were preferred to hand-tuned
parameters, confirming that the method can generate parameters suited for
different subjects.

A result of the offline optimization we use is that the method is largely
agnostic to the number of tunable parameters. Consequently, we were able to
additionally optimize neuromuscular model parameters for different speeds.  Our
experimental results show that these parameters improved the ankle work
characteristics and user preferences as speed increased for three out of four
tested parameter sets. 

Previous works such as \citet{eilenberg2010control} and
\citet{markowitz2011speed} have demonstrated control of powered ankle prostheses
via neuromuscular models of muscles and reflexes. This paper extends that work,
as it presents the first instance of neuromuscular control on a powered knee and
ankle prosthesis. With the tuning method presented here, we comprehensively test
this prosthesis control strategy in \cref{sec:nm_vs_imp}.

Due to the ability of the method to handle problems of arbitrary dimensionality
given sufficient computational resources, it may be used to tune many types of
control strategies. For example, one could use the approach to optimize the
phase variable control strategy, in which the prosthesis follows predefined knee
and ankle trajectories parameterized as functions of hip angle and hip angle
integral~\citep{quintero2016preliminary}. In this case, each bandit would
provide a different knee and ankle trajectory.  Importantly, the ability of this
method to optimize various types of controllers may help researchers compare
control strategies fairly, as we do in \cref{sec:nm_vs_imp}.

It is probable that the parameter library we obtained in this work can be
further improved. We propose two directions for further investigation: First, it
may be possible to improve the library by obtaining a larger set of gait data
and then using clustering algorithms to arrive at a reduced set of canonical
gaits. \citet{vardaxis1998classification} apply a similar idea to to EMG data to
cluster gaits into five major styles. With a library derived from canonical
gaits, we may cover the parameter space more evenly. A second approach is to use
biomechanical measurements from the amputee, such as segment lengths and
measured peak joint torques, to obtain probability distributions for the
neuromuscular model parameters. We could then compose the parameter library by
sampling parameters from the distributions and performing rigid body simulations
of the amputee and prosthesis system.

The optimization approach we have presented may have considerable practical
value for commercial prostheses as well. Because it uses preference feedback,
users of an assistive device can easily provide feedback via smartphones or
wearable devices. Moreover, the dueling bandit algorithm is well suited to
lifelong learning. Since the algorithm seeks to minimize regret, we can ensure
its exploration is only as obtrusive as necessary. 

The study we presented has several limitations. First, we only had five subjects
complete the study. Ideally, we would have more subjects than the number of
parameter sets so we could determine if any parameters are never preferred or if
any group is not currently well represented by the current set of parameters.
Also, we should confirm that the proposed optimization framework provides
suitable control parameters for amputees wearing this prosthesis.
