\section{Bayesian Approach Discussion}\label{sec:tuning_discussion}
We presented a new optimization algorithm (PES-P) that extends Predictive
Entropy Search to preference feedback. The algorithm addresses two key problems
frequently encountered in system optimization. First, it circumvents the often
difficult process of parameterizing and learning an objective function by
directly querying users for preferences between pairs of parameters. Second, the
algorithm minimizes the required number of experiments by employing Bayesian
optimization techniques that ensure the queries maximize the information gained
about the location of the optimum. Moreover, unlike previous approaches for
preference learning on robotic systems~\citep{wilson2012bayesian,
jain2013learning}, PES-P does not require a model of the system.

Our experiments show that the proposed algorithm outperforms baseline
algorithms. In most of the simulation experiments PES-P found optima that
achieved higher objective values than those found by the expected improvement
method (EI) or by random comparisons via Latin hypercubes (LH)
(\cref{fig:y_err_sim}). The reason why PES-P outperformed EI is likely due to
the former's explicit consideration of how the limited, noisy information
obtained from a preference query will affect the knowledge about potential
objective function optima. The acquisition function (\cref{eq:acquisition_orig})
recognizes that preferences become more uncertain the closer two sample points
are to each other. EI, on the other hand, does not reason about noisy
preferences and, instead, still assumes it can sample values
(\cref{eq:expected_improvement}). Consequently, EI ignores the distance between
sample points, which often leads to a greedy strategy that solicits preferences
between adjacent points. While this strategy can resemble gradient ascent with
convergence to local optima in a noise-free optimization, it often failed in our
experiments characterized by noisy observations. Note, however, that such
limitations were not observed by Brochu and colleagues~\citep{eric2008active},
who successfully used EI with preferences to optimize parameters for a graphics
application, possibly because the associated visual task produced less noisy
responses than did our simulations or prosthesis walking task. 

However, a major drawback of the proposed PES-P approach is its limited ability
to scale to problems of sufficient dimensionality. As shown in
\cref{fig:y_err_sim}d PES-P provides little benefit over random sampling on 5D
problems. In contrast, to optimize the neuromuscular model control proposed in
\cref{sec:nm_control_prosthesis} we need to be able to solve problems with
dozens of dimensions. Therefore, in the next half of this chapter, we explore an
alternative approach, that frames prosthesis optimization as a dueling bandits
problem \citep{yue2012k}.
