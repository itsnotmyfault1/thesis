\section{Challenges in Transfemoral Prosthesis
Control}\label{sec:intro_challenges} 

It still remains an open research question how best to control active prostheses
to achieve natural and robust gaits. Prosthesis controllers should address a
number of important challenges: 

\begin{challenges}
    \item\label{chal:robust} \emph{Control must ensure stability of prosthesis
    and amputee}~~\citet{miller2001prevalence} found that 49.2\% of lower
    limb amputees feared falling and that of those afraid of falls 76\% avoided
    physical activity as a result. Avoidance of physical activity is eminently
    concerning as it may lead to reduced strength, endurance, and balance,
    feeding a positive feedback loop that causes further debilitation.
    Therefore, to improve amputee quality of life it is imperative that powered
    prosthesis control strategies reduce the risk of falling. 
    
    In a traditional robotic system, this goal translates to ensuring the
    stability of the states of the robotic system. However, for a robotic
    prosthesis, the stability of the prosthesis state is insufficient to
    guarantee amputee stability. It is possible for all prosthesis states to be
    nominal while its user faces a precarious situation. Consequently,
    prosthesis controllers should ensure, either empirically or formally, that
    both the prosthesis and amputee states remain stable in the presence of
    myriad of disturbances. We investigate the robustness of proposed prosthesis
    controllers in
    \cref{sec:control_sim,sec:nm_vs_imp,sec:trip_avoidance,sec:phase_estimation}.
    
    \item\label{chal:sensing} \emph{Control should ideally only use information
    that can be garnered from sensors on the prosthesis itself}~~The above
    robust control robustness problem is complicated by sensing restrictions in
    a practical prosthetic device. While full instrumentation of the amputee's
    limbs and the prosthesis may help guarantee the stability of the overall
    system, in a practical prosthesis device, donning and doffing these sensors
    may be overly burdensome for amputees in the real world. Therefore, in this
    thesis, we only consider approaches that use information that could be
    reasonably garnered from sensors on the prosthesis and its socket.

    \item\label{chal:unique} \emph{Control must adapt to suit the needs of
    individual users}~~Variations in gait between amputees arise due to a number
    of factors including, but not limited to, the amputee's limb-lengths,
    weight, strength, endurance, reason for amputation, time since amputation,
    experience, and personal preferences. Consequently, prostheses and
    controllers should be optimized to suit individual users. In this thesis we
    also explore methods for optimizing prosthesis parameters from user
    preferences (\cref{sec:preference_optimization}) and in latter work propose
    methods that automatically adapt to the user's gait
    (\cref{sec:trip_avoidance,sec:phase_estimation}).

    \item\label{chal:dynamic} \emph{Control should allow the prosthesis to
    interact dynamically with the amputee and environment}~~In human walking,
    during stance the leg acts in a compliant, spring-like manner
    \citep{geyer2006compliant} and significant time is spent in
    statically-unstable contact on the heel or toe, suggesting the importance of
    mechanical stability achieved via foot placement \citep{perry2010gait}.
    During swing, ballistic motion explains much of the leg trajectory
    \citep{mochon1980ballistic}. Indeed, much of the entire gait cycle can be
    explained via passive dynamics as evidenced by passive-dynamic walkers that
    can stably walk down slight inclines with no onboard power
    source~\citep{mcgeer1990passive, collins2005efficient}.

    Consequently, in order to ensure that amputee gaits are natural, efficient,
    and robust to disturbances, it is essential that the design and control of
    robotic prostheses admit and leverage the inherent dynamics of walking.
    Therefore, in this thesis we detail the mechanical design and low-level
    control of a prosthesis that allows for precise torque tracking
    \cref{sec:pros_design}.  Furthermore, all controllers proposed in this
    thesis command desired torques to the prosthesis to allow for compliant
    interaction with the amputee and the environment. This is either
    accomplished via controllers that directly command torques as a function of
    sensor inputs, as in the case of neuromuscular control
    (\cref{sec:neuro_model}), or, when kinematic objectives are specified,
    accomplished via the use of feedforward torque commands and low-gain
    feedback, as in the case of our proposed swing (\cref{sec:trip_avoidance})
    and phase-based stance controls (\cref{sec:phase_estimation}).
\end{challenges}

In this thesis, we primarily focus on addressing \cref{chal:robust} by
evaluating the robustness of various prosthesis controllers, learning from those
evaluations and proposing new methods to improve gait stability. However, in all
presented work we make sure to keep \crefrange{chal:sensing}{chal:dynamic} in
mind.
