\section{Planning Approach Introduction}\label{sec:swing_control_planning_intro}
Lower limb amputees using state of the art commercial prostheses face a number
of gait deficiencies that negatively impact their quality of
life~\citep{gauthier1999enabling}. Of acute significance among these
deficiencies are the increased risk of falling and the related injuries, which
can lead to amputees avoiding activity out of a fear
falling~\citep{miller2001prevalence}.  As falls and their avoidance are linked
to swing leg placement in locomotion, active swing control strategies could help
to substantially reduce the risk of falling. However, current swing controllers
of transfemoral prostheses do little to proactively minimize this risk. 

Existing swing phase control approaches for powered prostheses predominantly
seek to reproduce the average swing phase behavior of the human leg. Whether the
approach is based on trajectory planning~\citep{lenzi2014speed}, impedance
control~\citep{sup2009preliminary}, or phase-based
control~\citep{quintero2016preliminary}, they all treat the swing phase motion
as an ``open loop'' problem with respect to trip hazards, as none of the
approaches take the location of the heel and toe of the prosthetic foot with
respect to the ground explicitly into account.  Therefore, current swing control
strategies neglect a clear advantage that robotic prostheses can have over their
passive counterparts: the ability to sense and act upon environmental
information.  Consequently, as shown in \cref{tab:treadmill_exp_fall_reasons},
trips during swing may pose a significant hazard to users of powered prostheses.

In this work, we develop a swing control strategy to reactively avoid trips with
powered transfemoral prostheses that uses visual information about the
environment and an estimate of the prosthesis configuration. Some previous
studies have explored incorporating visual feedback into the control of leg
prostheses. For example, \citet{scandaroli2009estimation} developed a state
estimator and controller that allowed the ankle joint of a prosthesis to conform
to the slope of the ground under the foot. To address the problem of terrain
recognition, \citet{zhang2011preliminary} developed a classifier using a LIDAR
and an IMU to discriminate between terrains such as flat ground and steps. More
recently, \citet{liu2016development} combined this terrain classifier with a
Bayesian intent classifier (based on \citep{du2012toward}) to develop an
environment-aware locomotion mode recognition system. In addition, RGBD sensors
have been explored as a source of rich environmental information for legged
assistance, including gait recognition \citep{massalin2017user} and stair
detection \citep{krausz2015depth,duan2018real}. However, none of these previous
studies have implemented a control strategy that uses environmental
information to reactively govern the motion of a powered prosthesis in
real-time. 

We present such a real-time reactive control of a powered prosthesis that
combines three building blocks. First, we use an extended Kalman filter (EKF)
that fuses measurements from an IMU, a LIDAR, and encoders on the prosthesis to
estimate the current pose of the prosthetic leg with respect to the ground.
Second, we predict likely future leg trajectories with sparse Gaussian process
models learned online during swing. Finally, we use the leg pose estimate and
trajectory predictions in a fast quadratic-program planner to reactively
generate in real time leg joint trajectories that avoid premature toe and heel
contact with the ground. To evaluate the proposed control, we compare our method
for trip avoidance to a standard non-reactive minimum-jerk trajectory planning
approach in a prosthesis walking experiment designed to elicit trips.
