\section{Planning Approach Discussion}\label{sec:swing_control_planning_discuss}

We presented initial work toward a real-time reactive control of powered
prostheses to help amputees avoid tripping in the swing phase of gait. At any
time during swing, the proposed control uses a laser range finder and an
inertial measurement unit to estimate the current pose of the prosthesis,
predicts the future hip angle and height based on trained Gaussian process
models, and then plans new knee and ankle joint trajectories that ensure neither
the toe nor heel contacts the ground prematurely. Our results indicate the
proposed control approach can substantially reduce the incidence of trips and
reduce the severity and frequency of toe scuffing.

To the best of our knowledge, this work is the first demonstration of lower limb
prosthetic control that integrates perception feedback in real-time and that
proactively ameliorates the falling hazard amputees face. Previous research in
this area has largely focused on detecting stumbles \emph{after} they have
occurred. For example, \citet{lawson2010stumble} and \citet{shirota2014recovery}
have proposed classifiers that can detect trips during swing and predict whether
a lower or raising strategy should be used in response. Similarly,
\citet{zhang2011towards} have proposed a method that can detect stumbles and
classify them as trips during swing or slips during stance. However, these
previous studies have not proposed concrete control actions to preempt stumbles
or to properly react in the event that a stumble is detected. Our results
motivate further research into such proactive and reactive approaches, closing
the perception-action loop for improving gait robustness with robotic
prostheses.

Several avenues for future work exist. First, in our current study only one
able-bodied user tested the proposed control. Further experiments with amputee
subjects are needed to verify the system provides benefits to this population.
For instance, amputees accustomed to walking with passive prostheses show
significantly altered hip kinematics~\citep{jaegers1995prosthetic}, which could
affect the control behavior. However, the proposed control should be able to
properly adapt to these behavior differences, as the Gaussian process models are
trained for specific users. Second, although trips during swing are one of the
most common failure modes we encounter with our powered prostheses, these events
are still rare and many hours of normal walking are required to observe a
sufficient number of trips and compare controllers. As a result, we actively
induced trips by sudden drops in hip height during swing, which does not exactly
reflect the situations in which trips occur.  Specifically, trips can happen due
to subtle changes in leg kinematics, and it remains to be seen in experiments if
our approach can avoid trips in these more subtle situations.

At the implementation level, there is also room for further exploration. To keep
the  computational costs low, and due to ease of implementation in the
prosthesis' Simulink Real-Time\textsuperscript{TM} environment, we plan
trajectories using quadratic programs that iterate between finding solutions for
the ankle and knee joints.  While this iterative approach is fast when compared
to trajectory optimization methods that deal with multiple joints
simultaneously, the iterations occasionally get stuck when the planner for one
joint trajectory cannot find a solution based on the assumed fixed trajectory of
the other joint. Moreover, if a solution cannot be found, the current approach
simply reuses the last identified trajectory, rather than moving the trajectory
to be more safe, even if it cannot fully satisfy the bounds. It seems worthwhile
to investigate whether non-convex trajectory optimization methods such as CHOMP
\citep{ratliff2009chomp}, in which the bounds are represented as soft rather
than hard constraints, can help solve for the knee and ankle trajectories
simultaneously without sacrificing computational speed.

In addition, several technical simplifications can be considered to bring this
technology closer to commercialization. We used an accurate and expensive laser
distance sensor, eyeing future research in obstacle scanning and avoidance
capabilities. However, for simple ground plane avoidance, inexpensive infrared
distance sensors such as those used by \citet{scandaroli2009estimation} are
likely sufficient. It may also be possible to simplify the trajectory planning
phase by, for example, forgoing formal guarantees on satisfying bounds and
instead relying on heuristics to increase knee and ankle flexion and adjust
timing in response to decreased hip height during swing.
