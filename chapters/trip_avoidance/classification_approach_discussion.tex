\section{Classification Approach Discussion}

We developed an online learning system to help users of powered transfemoral
prostheses avoid obstacles. Our system uses information from an inertial
measurement unit during the late stance to early swing period to classify the
upcoming swing as either normal or a trip avoidance attempt. Unlike previous
work on obstacle negotiation for transfemoral prostheses
\citep{lawson2010stumble, zhang2011towards, shirota2014recovery}, our system
learns online on an actual transfemoral prostheses. We compared the
classification performance of our online system with a hypothetical offline
system using online trials to provide testing and training data for offline
analysis. This comparison showed that the online learning system provided an
improvement in sensitivity and accuracy to obstacle avoidance attempts. Both an
experienced, able-bodied subject and an inexperienced, amputee subject were able
to improve their obstacle avoidance success rates significantly. However, only
the experienced, able-bodied subject was able to achieve some level of
volitional control of the prosthesis flexion as a function of obstacle height.
                                  
There are several reasons why the amputee subject may not have been able to
achieve volitional control of prosthesis flexion. First, the amputee had far
less experience using the prosthesis than the able-bodied subject. Consequently,
even though both subjects were informed that trying harder to lift the leg over
bigger obstacles would likely lead to greater flexion once the prosthesis
learns, it is likely that only the first subject was able to incorporate and
implement this information. The amputee may have concentrated on more
rudimentary aspects of gait, as evidenced by his use of the handrails to walk,
whereas the able-bodied subject did not need to use the hand rails. Moreover,
the amputee's socket may have provided less control over the prosthesis than did
the intact subject's able-bodied adapter (shown in \cref{fig:prosthesis}).
Finally, we noted that the relationship between joint flexion and obstacle
height tended to oscillate over the course of our trials. This may imply that
the gains we used for the target knee angle regression (\cref{eq:tgt_angle})
were too high.

Before settling on the specifics of the obstacle avoidance system presented
here, we also tested other options for its components. For example, we also
evaluated incorporating EMG signals from the non-prosthetic limb in our obstacle
avoidance classifier. Previous research showing that able-bodied subjects
utilize stance leg musculature to help raise the hip during obstacle avoidance
motivated this choice of EMG placement \citep{patla1995role}. However, as was
found by \citet{spanias2018online}, using EMG data along with mechanical data in
the forwards-backwards online learning algorithm did not seem to improve
classification accuracy, which is already high. This lack of improvement may
also result from a significant delay in our wireless EMG sensors (Delsys
Trigno). It is possible that a low-latency wired EMG sensor would be able to
improve classification performance or the performance of the target angle
regression.

We also tried using imitation learning techniques to model able-bodied
strategies for stepping over obstacles. Specifically, we employed maximum margin
inverse optimal control \citep{ratliff2007online} to learn, offline, cost
functions for the knee that explained obstacle avoidance trajectories. However,
when used online, the generated trajectories tended to diverge and produce
unexpected results because the initial toe-off state of the prosthesis did not
match those in the able-bodied data set. For the obstacle avoidance classifier,
we correct this sort of offline-online mismatch via the backwards classifier
that provides labels to train the forwards classifier online. It is less clear
how to update trajectories in hindsight as we never see the obstacle. For this
reason, we used bang-bang trajectories during obstacle avoidance, which maximize
the time the joints remain flexed.

In the future, we plan to overcome this issue by incorporating a laser distance
sensor into the prosthesis. This sensor should allow precise measurement of the
ground and obstacle shape during the initial part of swing as the hip moves
forward. We plan to then use this information to explicitly plan knee and ankle
trajectories that will avoid the obstacle and the floor until the appropriate
touch down time.

There are several other limitations of the current study we should address in
future work as well. First, we only tested the algorithm with two subjects. More
subjects of varying skill levels are necessary to determine how applicable the
system is to a broader population. Additionally, a likely reason why the forward
classifier's sensitivity was relatively low, was that there were many more
normal steps than obstacle avoidance attempts in the training data set. This may
cause the SVM loss function's minimum to focus more heavily on classifying
normal steps correctly. Deploying this system on a commercial prosthesis, for
which trips are more rare, would exacerbate this issue. Therefore, future
development should investigate how to train a classifier given heavily
unbalanced class frequencies.
