\section{Classification Approach Introduction}\label{sec:swing_control_class}

\begin{figure*}[t]
\centering
\includegraphics[width=\textwidth]{avoid_frames}
\caption{a)~Utilizing minimum jerk trajectories during swing does not allow for
appropriate adaptation of swing trajectories to enable obstacle avoidance.
b)~Our adaptive system learns online to detect the presence of an obstacle from
the amputee's late stance/early swing movements. Once detected, the controller
modifies the trajectories of the knee and ankle to achieve improved obstacle
clearance.}
\label{fig:avoid_frames}
\end{figure*}

Avoiding obstacles on the ground is a necessity for maintaining safety while
performing a variety of locomotion tasks. This behavior requires anticipation of
an obstacle and active leg control strategies to avoid it \citep{patla1995role}.
Transfemoral amputees, however, have a compromised ability to negotiate
obstacles, as shown in \Cref{fig:avoid_frames}, as current prosthesis technology
relies on mechanically passive knees that necessitate significant compensation
at the hip in order to replicate able-bodied trip recovery strategies
\citep{shirota2015transfemoral}. Compromised ability to avoid and recover from
trips may contribute to the large number of falls that leg amputees suffer. For
instance, 58\% of unilateral amputees reported a fall within a year
\citep{kulkarni1996falls}. Moreover, the fear of falling can cause amputees to
avoid activity, leading to further deterioration of their physical condition
\citep{miller2001prevalence}.

An increasing availability of powered prostheses in research labs provides the
opportunity to study active obstacle avoidance strategies in prosthetics,
although so far only a limited number of studies exist on this topic. These
studies focus on detecting and classifying the correct response strategy after
the amputee has tripped. For example, \citet{lawson2010stumble} developed a
classifier that uses fast Fourier transform and the root mean square of
accelerometer data as features to classify stumbles and recovery strategies,
respectively. \citet{zhang2011towards} found that adding EMG signals from the
residual limb to accelerometer data can help reduce false positives for stumble
and strategy detection. Finally, \citet{shirota2014recovery} identified the
optimal sliding window lengths and increments for feature calculation for trip
detection and strategy selection classifiers. While detecting and classifying
trip recovery strategies after their occurrence is a necessary step towards
obstacle avoidance, it does not provide a proactive prosthesis control strategy
that prevents obstacle encounters in the first place.

Another major drawback of the previous studies is that they train and test 
classifiers offline. However, a deployed trip classifier needs to function
online and deal with temporal adaptation of the learner and amputee. The
adaptation is required as the obstacle avoidance behavior triggered by a trip
classifier alters the amputee's movements and, therefore, the data used to train
the classifier. Consequently, trip classifiers trained offline may be
ineffective due to a mismatch of training and testing data. In
\cref{sec:back_high_level_control} we reviewed high-level classifiers that
detect transitions between gait modes such as standing, level ground walking,
and stair/ramp ascent/descent. In that setting, classification approaches often
run into a similar problem of training and testing data mismatch.
\citet{spanias2018online} provides a method of rectifying this data distribution
mismatch.

Here we present the first pilot study that combines online learning and
proactive control of a powered transfemoral prosthesis to implement obstacle
avoidance in amputee locomotion. The obstacle avoidance system uses early-swing
measurements of the residual limb angle, angular velocity, and linear
acceleration to recognize in-process obstacle avoidance attempts. To address the
online learning aspect of this system, we adapted the method proposed by
\citet{spanias2018online} for online learning of gait mode classification. We
also changed the existing swing leg behavior of the prosthesis to facilitate
obstacle avoidance. This change includes a regression to predict the appropriate
degree of knee and ankle flexion given the user's previous obstacle response
motions. Finally, we evaluated the system behavior in trials with both
non-amputee and amputee subjects.
