\section{Stance Reflexes}\label{sec:neuro_stance_reflexes}

During stance, hypothesized reflex feedback pathways stimulate the muscles of
the leg. In general, a linear feedback law governs the stimulation
$\func{S}[][m]{t}$ of muscle $m$, 
\begin{align}
    \func{S}[][m]{t} = S_0^m + \sum_n G_n^m \func{Pro}[n][]{t - \Delta t_n},
    \label{eq:lin_reflex}
\end{align}
where $S_0^m$ is a constant pre-stimulation, $\func{Pro}[n][]{t - \Delta t_n}$
is the time-delayed proprioceptive signal from muscle $n$, and $G_n^m$ is the
gain on that signal. The proprioceptive signal can take the form of force
feedback, $\func{F}[\tn{n}][m]{\cdot}$, which uses the time delayed tendon
force, or length feedback, $\func{L}[\tn{n}][m]{\cdot} =
\func{l}[n][\tn{CE}]{\cdot} - ^\tn{off}l_{n}^{m}$, which uses the difference
between the length of the contractile element of muscle $n$ and an offset length
$^\tn{off}l_{n}^{m}$.

The time delay we apply to proprioceptive signals estimate the round-trip neural
signal transmission delay of afferent signals from muscle spindles and Golgi
tendons to the spine and efferent signals back to the muscles. For ankle
muscles, the soleus, tibialis anterior, and gastrocnemius, the time delay is
$\Delta t_n = \unit[20]{ms}$. For knee muscles, the vastus and hamstrings, it is
$\Delta t_n = \unit[10]{ms}$. For the hip muscles, the hamstrings, gluteus, and
hip flexors, the time delay is $\Delta t_n = \unit[5]{ms}$. We will denote time
delayed signals using $t_\tn{l} = t - \unit[20]{ms}$, $t_\tn{m} = t -
\unit[10]{ms}$, and $t_\tn{s} = t - \unit[5]{ms}$.


The reflexes encode several key functions of legged locomotion: generating
compliant leg behavior, preventing knee overextension, and balancing the trunk.
The first function, generating compliant leg behavior, is achieved via positive
force feedback on the monoarticular leg extensors: the soleus, vastus, and
gluteus. For example, the reflexes stimulate the vastus in part by
\begin{align}
    \func{S}[][\tn{vas}]{t} = S_0^\tn{vas} + 
        G_\tn{vas}^\tn{vas} \func{F}[\tn{vas}][]{t_\tn{m}} + \ldots \:.
\end{align}

To implement the second function, preventing knee overextension, the reflex
control uses two strategies. First, positive force feedback of the biarticular
gastrocnemius and hamstrings muscles helps counteract the tendency for knee
overextension caused by ankle plantarflexion and hip extension torques
respectively. For example, the gastrocnemius has a force feedback reflex,
\begin{align}
    \func{S}[][\tn{gas}]{t} = S_0^\tn{gas} + 
        G_\tn{gas}^\tn{gas} \func{F}[\tn{gas}][]{t_\tn{l}},
\end{align}
that flexes the knee as it contributes to ankle plantarflexion. The hamstring
also has a positive force feed back
\begin{align}
    \func{S}[][\tn{ham}]{t} = S_0^\tn{ham} + 
        G_\tn{ham}^\tn{ham} \func{F}[\tn{ham}][]{t_\tn{s}} + \ldots
        \label{eq:ham_stim}
\end{align}
that counteracts knee extension caused by hip extension. Also, the hamstring
force feedback helps prevent hip flexion caused by heel-strike.

A second mechanism further protects the knee by inhibiting the vastus
stimulation in proportion to knee extension beyond a threshold, resulting in
the complete vastus stimulation
\begin{fullwidth}
\begin{align}
    \func{S}[][\tn{vas}]{t} &= S_0^\tn{vas} + 
        G_\tn{vas}^\tn{vas} \func{F}[\tn{vas}][]{t_\tn{m}} 
            -G_\tn{knee}^\tn{vas} \left(\phi_\tn{knee}(t_\tn{m}) -
            ^\tn{off}\phi_\tn{knee} \right)
            \left( \phi_\tn{knee}(t_\tn{m}) < ^\tn{off}\phi_\tn{knee} \right)
            \left( \dot \phi_\tn{knee}(t_\tn{m}) < 0\right)
\end{align}
\end{fullwidth}
where $^\tn{off}\phi_\tn{knee}$ is the angle beyond which the vastus is inhibited.

The reflexes achieve the final function of balancing the trunk by
proportional-derivative control that produces stimulations for the hip muscles
(hip flexors, gluteus, and hamstrings) to stabilize the trunk at a reference
lean. Because muscles can only provide pulling force, the proportional derivative
control signal is distributed as hip flexor stimulation if the signal represents
flexion torque and as simultaneous stimulation for the gluteus and hamstrings if
it represents hip extension torque. For example, the complete hamstrings
stimulation becomes
\begin{align}
    \func{S}[][\tn{ham}]{t} &= S_0^\tn{ham} + 
        G_\tn{ham}^\tn{ham} \func{F}[\tn{ham}][]{t_\tn{s}} \notag\\
            &\quad +  \left\{ G_\tn{p}^\tn{ham} 
            (\phi_\tn{trunk}(t_\tn{s}) - \phi_\tn{trunk}^\tn{ref}) 
            + G_\tn{d}^\tn{ham} \dot{\phi}_{\tn{trunk}}(t_\tn{s}) \right\}_+ 
\end{align}
where the third term returns the positive reflex contribution from the trunk
balance control. 

\begin{fullwidth}
The full set of stance reflexes are:
\begin{align}    
    \func{S}[][\tn{sol}]{t} &= S_0^\tn{sol} + 
        G_\tn{sol}^\tn{sol} \func{F}[\tn{sol}][]{t_\tn{l}}
        \label{eq:sol_stim}\\
    \func{S}[][\tn{ta}]{t} &= S_0^\tn{ta} + 
        G_\tn{ta}^\tn{ta} \func{L}[\tn{ta}][]{t_\tn{l}} - G_\tn{sol}^\tn{ta} 
        \func{F}[\tn{sol}][]{t_\tn{l}} \label{eq:ta_stim}\\
    \func{S}[][\tn{gas}]{t} &= S_0^\tn{gas} + 
        G_\tn{gas}^\tn{gas} \func{F}[\tn{gas}][]{t_\tn{l}} 
        \label{eq:gas_stim_full}\\
    \func{S}[][\tn{vas}]{t} &= S_0^\tn{vas} + 
        G_\tn{vas}^\tn{vas} \func{F}[\tn{vas}][]{t_\tn{m}} 
        -G_\tn{knee}^\tn{vas}\left( \phi_\tn{knee}(t_\tn{m}) -
        ^\tn{off}\phi_\tn{knee} \right) 
        \left( \phi_\tn{knee}(t_\tn{m}) < ^\tn{off}\phi_\tn{knee} \right)
        \left( \dot \phi_\tn{knee}(t_\tn{m}) < 0\right)    
        \label{eq:vas_stim_full}\\
    \func{S}[][\tn{ham}]{t} &= S_0^\tn{ham} + 
        G_\tn{ham}^\tn{ham} \func{F}[\tn{ham}][]{t_{s}} 
        + \left\{ G_\tn{p}^\tn{ham} (\phi_{\tn{trunk}} - \phi_\tn{trunk}^\tn{ref}) 
        + G_\tn{d}^\tn{ham} \dot{\phi}_{\tn{trunk}} \right\}_+
        \label{eq:ham_stim_full} \\
    \func{S}[][\tn{glu}]{t} &= S_0^\tn{glu} + 
        G_\tn{glu}^\tn{glu} \func{F}[\tn{glu}][]{t_{s}}
        + \left\{ G_\tn{p}^\tn{glu} (\phi_{\tn{trunk}} - \phi_\tn{trunk}^\tn{ref}) 
        + G_\tn{d}^\tn{glu} \dot{\phi}_{\tn{trunk}} \right\}_+  \\
    \func{S}[][\tn{hfl}]{t} &= S_0^\tn{hfl} + 
        \left\{ G_\tn{p}^\tn{hfl} (\phi_{\tn{trunk}} - \phi_\tn{trunk}^\tn{ref}) 
        + G_\tn{d}^\tn{hfl} \dot{\phi}_{\tn{trunk}} \right\}_- 
\end{align}    
\end{fullwidth}
