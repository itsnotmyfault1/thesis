\chapter{Conclusion}\label{sec:thesis_conclusion}

This thesis began with the goal of improving the robustness of gait with
transfemoral prostheses so as to reduce amputees risk of falling and increase
their quality of life. Towards the goal, we started by comparing a proposed
neuromuscular strategy for transfemoral prosthesis control to the established
impedance control strategy. We attempted to make this comparison as objective as
possible by building a robotic prosthesis capable of accurate torque control
(\cref{sec:pros_design}) and developing methods that use the user's feedback to
select prosthesis control parameters (\cref{sec:preference_optimization}). While
the experiment comparing neuromuscular and impedance control did not confirm our
hypothesis that neuromuscular control would lead to a significant reduction in
falls (\cref{sec:nm_vs_imp}), it did clarify the importance of state estimation
in prosthesis control. This insight motivated the development of swing and
stance controllers with state estimation via an extended Kalman filter (EKF) at
the core of each (\cref{sec:trip_avoidance,sec:phase_estimation}). The proposed
swing controller uses an EKF to estimate the position and orientation of the
hip, which is then used to plan knee and ankle swing trajectories that avoid
trips. The proposed stance controller uses an EKF to estimate the phase of
stance. This phase estimate made the gait robust to ground height disturbances
and able to adapt to both sudden and gradual changes in gait speed. 

In performing the work in this thesis we have assembled a substantial amount of
practical knowledge and suggestions for future research in the field. First,
when it comes to prosthesis design, in hindsight, we likely put too much
emphasis on prosthesis performance, and did not consider enough whether the user
would actually be able to make use of the performance. In our experiments,
able-bodied users wore the prosthesis through a L-shaped adapter
(\cref{fig:prosthesis_actual}) while amputees used their own socket
(\cref{fig:prosthesis_phase_exp}). Both of these interfaces are not very secure,
limiting the weight of the prosthesis and the dynamism of movements. Therefore,
while our prosthesis design, at \unit[6]{kg}, is of comparable mass to a
biological limb, it is too heavy to use comfortably for an extended period of
time. Future prosthesis designs should be mindful of the constraint the
interface places on useable torque and comfortable prosthesis mass. One solution
for a transfemoral prostheses design for research purposes, which achieves high
performance and keeps weight to a minimum, is to create an knee-ankle prosthesis
emulator, similar to the ankle emulator created by
\citet{caputo2013experimental}.

A difficulty faced when conducting this research was the substantial time spent
implementing various controllers from the literature on our transfemoral
prosthesis prototype.  This challenge would be significantly reduced with
standardization of the interface between the low-level controls and the mid
level behavior controls. Standardizing this interface would more easily allow
researchers to share code and try controllers on their own prostheses.
Standardizing the hardware design too would also help in this effort and
substantially decrease the time required to make progress in this field. The
recent development of open source knee and ankle prosthesis designs should help
significantly in this regard \citep{azocar2018design}.

Another development that would help decrease the time required to improve
prosthesis controls is better simulations of amputees and their interactions
with hardware. In \cref{sec:control_sim}, we presented results from one such
simulation that predicted more falls for impedance control than neuromuscular
control when walking on rough ground. Based on the experimental results on the
real prosthesis that we have gathered, it seems these simulations may have
overstated the potential improvement in gait robustness offered by neuromuscular
control. We may be able to improve these sorts of simulations to better predict
real-world results in two ways. First, through the course of conducting this
research, we have recorded large amounts of motion capture data of users walking
on the prosthesis. This data could be used to improve the neuromuscular model to
better capture actual subject responses to disturbances. Second, in
\cref{sec:control_sim}, we used the shooting method and a genetic algorithm
\citep{hansen2006cma} to optimize the control parameters of the prosthesis.
These optimizations often required days to converge on a solution. In future
work, more advanced simulation/trajectory optimization techniques should be
considered such as using symbolic differentiation to derive the change in the
control action with respect to parameters and using nonlinear programming and
collocation to directly solve the optimal control problem
\citep{hargraves1987direct}. 

Finally, an innovation provided by the planning-based swing control described in
\crefrange{sec:swing_control_planning_intro}{sec:swing_control_planning_discuss}
is that it considers environmental information in real-time within the mid-level
controller. The environmental information directly and precisely impacts the
trajectories that are generated during swing. In contrast, previous prosthesis
controllers that have considered environmental information have primarily only
done so at the high-level for gait mode recognition (see
\cref{sec:back_high_level_control}). These high-level strategies switch the
mid-level control parameters within a discrete set of options. Similarly, the
phase-based stance control described in
\crefrange{sec:swing_control_planning_intro}{sec:swing_control_planning_discuss}
can use information in real-time from multiple sensors on the prosthesis to
reason about the appropriate control action to take. These results motivate
further research into mid-level controllers that can directly act on more
sources of information. Doing so may allow prostheses to adapt more
appropriately to the user and environment and take a broader range of actions
that are better suited to the specific situation at hand. 
