\section{Prosthesis Optimization}\label{sec:back_optimization}

Optimizing control policies for prostheses and exoskeletons presents a
challenging task due to four key issues. First, there is significant
variability gait characteristics that precludes using the same parameters for
all users. Recently, \citet{zhang2017human} demonstrated this by optimizing an
ankle exoskeleton's torque trajectory for specific users.  The authors found
that optimized torque trajectories could reduce metabolic energy consumption
beyond that provided by a generic assistance strategy.
 
Second, to optimize prostheses and exoskeletons, it is necessary to define an
objective function that includes and correctly assigns importance to all
characteristics that determine system performance. Most prior work in this area
assumes a certain form for the objective function. For example, in the work
described above, the author's assume that the metabolic energy consumption of
the user is the only important factor. The authors then use the Covariance
Matrix Evolution Strategy \citep{hansen2006cma} to optimize the parameters of
the exoskeleton. In other work, \citet{huang2016cyber} optimize a transfemoral
leg prosthesis assuming the objective is to improve the ability of the knee to
track an able-bodied trajectory. In this work, the authors use a
\emph{cyber-expert system} (CES) that encodes how a human expert's tuning of
impedance parameters effects the trajectory error. With this strategy, the
author's improved trajectory tracking of a knee prosthesis, but also note that
other metrics might be important such as metabolic energy, symmetry, and
disturbance rejection. This approach was later improved by using adaptive
reinforcement learning to circumvent predefining the tuning rules
\citep{wen2016adaptive,wen2019online}. 

If we choose to optimize more than one outcome simultaneously, we need to
assign weights to each feature to each feature that reflect the users individual
needs. Moreover, other aspects of gait may also be important but difficult to
quantify, such as the amputee's comfort and sense of stability. In any case,
measuring gait features require a high level of technical expertise and
equipment to measure, and therefore preclude an amputee tuning his or her own
prosthesis.

To solve the problem of defining and measuring objective functions for robotic
systems that human operators can directly control, researchers have proposed
\emph{learning from demonstration} (LfD)~\citep{argall2009survey}. In this
paradigm, one can either circumvent learning the objective function by directly
learning a policy that matches the distribution of state-action pairs recorded
during human demonstrations of the desired
behavior~\citep{pomerleau1991efficient, schaal1999imitation}, or one can learn a
reward function consistent with the demonstrator's actions and visited states
and use it to derive an optimal control~\citep{ng2000algorithms,
ratliff2006maximum, ziebart2009human}. LfD methods are attractive because they
allow non-experts to specify both the quantifiable and qualitative aspects of
the desired robot behavior via the non-technical language of demonstration.

For robot behavior that people cannot demonstrate, such as the optimal behavior
of an amputee's prosthesis, or the desired behavior of complex, dynamic robots,
we can alternatively query human users for qualitative feedback in order to
shape the robot policy. For example, the TAMER
framework~\citep{knox2009interactively, knox2013training} utilizes good/bad
assessments of a robot's recent actions to optimize its policy.
\citeauthor{pilarski2011online} use this method to allow subjects to optimize
the policy of an EMG-controlled prosthesis arm via their positive and negative
feedback signals~\citep{pilarski2011online}.  Another paradigm in qualitative
feedback is to obtain \emph{preference feedback} between two or more policies or
sequences of actions, which may provide more nuanced feedback than absolute
ratings. For example, \citeauthor{jain2013learning} and
\citeauthor{akrour2014programming} propose methods that learn a user's
trajectory scoring function based on his rankings of possible
policies~\citep{jain2013learning, akrour2014programming}. Similarly,
\citeauthor{wilson2012bayesian} provide a method to directly identify a user's
preferred policy based on her preferences between pairs of demonstrated
trajectories~\citep{wilson2012bayesian}. These prior works demonstrate that we
may be able to use qualitative feedback, such as preferences, from non-expert
users to program robot behavior, without prescribing an objective function.
However, a drawback of the aforementioned methods that learn from preference
feedback is their reliance on simulators to predict system behavior.
Human-in-the-loop systems, such as lower-limb prostheses and exoskeletons, are
challenging to simulate accurately, making these methods difficult to apply.

The third issue an operator tasked with optimizing control policies for
human-in-the-loop systems faces is the expense, in terms of time and effort, of
repeatedly executing policies. Consequently, stochastic sampling approaches may
be less applicable in this domain. To minimize the number of trials needed,
researchers have proposed black-box \emph{Bayesian Optimization} (BO) methods
that model both the objective function and its uncertainty. In these methods,
the uncertainty informs an acquisition function that speeds up the optimization
by exploiting regions of the parameter space with believed high objective value
while still exploring regions where the objective function is uncertain. For
example, researchers have successfully employed BO methods to efficiently
optimize the gait parameters of a robotic snake~\citep{tesch2011using} and a
dynamic bipedal robot~\citep{calandra2014bayesian}. In
\cref{sec:preference_optimization} we present a new bayesian optimization method
that uses learning from preferences between pairs of control parameters to avoid
priori definition of features and to consider unquantifiable qualities of the
desired behavior. We apply this method to optimize several simulated tasks.

However, we also find through these experiments that the proposed Bayesian
optimization approach, cannot scale to the dimensionality required for
prosthesis optimization. This reveals the fourth challenge of prosthesis
optimization, which is that prosthesis controllers typically have dozens of
parameters, causing optimization routines to suffer from the curse of
dimensionality. Therefore, in \cref{sec:preference_optimization} we also explore
using a dueling bandits \citep{yue2012k} approach to optimizing the prosthesis
parameters. This approach uses significant offline computation to generate a
discrete library of viable parameter sets from which the user can choose.
