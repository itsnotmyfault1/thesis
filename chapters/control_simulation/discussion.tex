\subsection{Discussion}\label{sec:completed_comparison_discuss}

Our simulation results suggest that the hybrid neuromuscular control policy may
be able to improve amputee gait stability over existing impedance control
methods. An amputee model walking with a powered prosthesis showed substantial
improvements in balance recovery on rough ground and after swing leg trips when
using the hybrid neuromuscular control policy as opposed to impedance control.
One possible reason for the improvement is that the proposed controller
considers global leg information such as the target leg angle
(\crefrange{eq:flexphase}{eq:stop}), and it is well known that without placing
the feet into proper target points on the ground, legged systems fail to balance
\citep{townsend1985biped,raibert1986legged,kajita20013d,
seyfarth2002movement,pratt2006capture,wu20133}. A second reason could be that
the design of the swing leg control policy explicitly accounts for large
disturbances to the lower limb dynamics in order to achieve desired leg
placements \citep{desai2012robust}. Finally, the implemented impedance control
strategy relies on reliable estimation of the discrete phase of gait during
stance so that the it can apply the appropriate impedance control parameters. In
the presense disturbances such as the ground height disturbances we studied
here, this phase estimate may be incorrect, thus causing the impedance control
to apply inappropriate torques.  

These results capture only a small portion of the balance disturbances that
humans typically encounter \citep{robinovitch2013video}. Other disturbances may
evoke amputee responses that the simulation model does not capture; especially
since it is driven solely by a reflexive walking controller that ignores
conscious interventions. Therefore, in future chapters of this thesis, we build
towards and present results wherin we implement prosthesis controllers on real
hardware being used by actual humans in order to more decisively compare control
strategies.
