%%%%%%%%%%%%%%%%%%%%%%%%%%%%%%%%%%%%%%%%%%%%%%%%%%%%%%%%%%%%%%%%%%%%%%
% How to use writeLaTeX: 
%
% You edit the source code here on the left, and the preview on the
% right shows you the result within a few seconds.
%
% Bookmark this page and share the URL with your co-authors. They can
% edit at the same time!
%
% You can upload figures, bibliographies, custom classes and
% styles using the files menu.
%
% If you're new to LaTeX, the wikibook is a great place to start:
% http://en.wikibooks.org/wiki/LaTeX
%
%%%%%%%%%%%%%%%%%%%%%%%%%%%%%%%%%%%%%%%%%%%%%%%%%%%%%%%%%%%%%%%%%%%%%%
\documentclass{tufte-book}

\makeatletter
\renewcommand{\subsubsection}{\ttl@straightclass{subsubsection}}
\makeatother
\titleformat{\subsubsection}[hang]
  {\em}
  {\thesubsubsection}
  {1em}
  {}
  []
\setcounter{secnumdepth}{2}
\setcounter{tocdepth}{2}

%\hypersetup{colorlinks}% uncomment this line if you prefer colored hyperlinks (e.g., for onscreen viewing)


%%
% If they're installed, use Bergamo and Chantilly from www.fontsite.com.
% They're clones of Bembo and Gill Sans, respectively.
%\IfFileExists{bergamo.sty}{\usepackage[osf]{bergamo}}{}% Bembo
%\IfFileExists{chantill.sty}{\usepackage{chantill}}{}% Gill Sans

\usepackage{microtype}
\usepackage[utf8]{inputenc}
%%
% For nicely typeset tabular material
\usepackage{booktabs}

%%
% For graphics / images
\usepackage{graphicx}
\setkeys{Gin}{width=\linewidth,totalheight=\textheight,keepaspectratio}
\graphicspath{{./figures/}}

% The fancyvrb package lets us customize the formatting of verbatim
% environments.  We use a slightly smaller font.
\usepackage{fancyvrb}
\fvset{fontsize=\normalsize}

\usepackage{amsmath} % assumes amsmath package installed
\usepackage{amssymb}  % assumes amsmath package installed
\usepackage{bm} %bold italic vectors
\usepackage{algorithm}
\usepackage{algpseudocode}
\usepackage{units}
\DeclareMathOperator{\argmin}{argmin}

%%
% Prints argument within hanging parentheses (i.e., parentheses that take
% up no horizontal space).  Useful in tabular environments.
\newcommand{\hangp}[1]{\makebox[0pt][r]{(}#1\makebox[0pt][l]{)}}

%%
% Prints an asterisk that takes up no horizontal space.
% Useful in tabular environments.
\newcommand{\hangstar}{\makebox[0pt][l]{*}}

%%
% Prints a trailing space in a smart way.
\usepackage{xspace}

%%
% Some shortcuts for Tufte's book titles.  The lowercase commands will
% produce the initials of the book title in italics.  The all-caps commands
% will print out the full title of the book in italics.

% Prints the month name (e.g., January) and the year (e.g., 2008)
\newcommand{\monthyear}{%
  \ifcase\month\or January\or February\or March\or April\or May\or June\or
  July\or August\or September\or October\or November\or
  December\fi\space\number\year
}

% Prints an epigraph and speaker in sans serif, all-caps type.
\newcommand{\openepigraph}[2]{%
  %\sffamily\fontsize{14}{16}\selectfont
  \begin{fullwidth}
  \sffamily\large
  \begin{doublespace}
  \noindent\allcaps{#1}\\% epigraph
  \noindent\allcaps{#2}% author
  \end{doublespace}
  \end{fullwidth}
}

% Inserts a blank page
\newcommand{\blankpage}{\newpage\hbox{}\thispagestyle{empty}\newpage}

% Macros for typesetting the documentation
\newcommand{\hlred}[1]{\textcolor{Maroon}{#1}}% prints in red
\newcommand{\hangleft}[1]{\makebox[0pt][r]{#1}}
\newcommand{\hairsp}{\hspace{1pt}}% hair space
\newcommand{\hquad}{\hskip0.5em\relax}% half quad space
\newcommand{\TODO}{\textcolor{red}{\bf TODO!}\xspace}
\newcommand{\ie}{\textit{i.\hairsp{}e.}\xspace}
\newcommand{\eg}{\textit{e.\hairsp{}g.}\xspace}
\newcommand{\na}{\quad--}% used in tables for N/A cells

%my commands
\synctex=1
\usepackage{comment}
\usepackage{todonotes}
\usepackage{cleveref}
\newcommand{\creflastconjunction}{, and\nobreakspace}
\usepackage{caption}
\usepackage{subcaption}
\captionsetup{compatibility=false}

%list types
\usepackage{enumitem}
\newlist{steps}{enumerate}{10}
\setlist[steps]{label*=\arabic*., ref=\arabic*}
\crefname{stepsi}{step}{steps}
\Crefname{stepsi}{Step}{Steps}

\newlist{challenges}{enumerate}{10}
\setlist[challenges]{label=\emph{Challenge \arabic*:}, ref=\arabic*, align=left, 
    listparindent=\parindent, parsep=0pt}
\crefname{challengesi}{challenge}{challenges}
\Crefname{challengesi}{Challenge}{Challenges}

\newlist{contributions}{enumerate}{10}
\setlist[contributions]{label=\emph{Contribution \arabic*:}, ref=\arabic*, align=left, 
    listparindent=\parindent, parsep=0pt}
\crefname{contributionsi}{contribution}{contributions}
\Crefname{contributionsi}{Contribution}{Contributions}

\newlist{tasks}{enumerate}{10}
\setlist[tasks]{label=\emph{Task \arabic*:}, ref=\arabic*,
    align=left, listparindent=\parindent, parsep=0pt}
\crefname{tasksi}{task}{tasks}
\Crefname{tasksi}{Task}{Tasks}

\newcommand{\proposaltitle}{Design and Evaluation of Robust Control Methods for
Robotic Transfemoral Prostheses}

\newcommand{\prob}[1]{\operatorname{P} \left( #1 \right)}
\newcommand{\tn}[1]{\mathrm{#1}}

\usepackage{xparse}
\DeclareDocumentCommand{\func}{m o o m}
{%
    \IfValueTF{#2}
        {%
        \IfValueTF{#3}
            {\operatorname{#1}_{#2}^{#3} \left( #4 \right)}
            {\operatorname{#1}_{#2}      \left( #4 \right)}
        }
        {%
        \IfValueTF{#3}
            {\operatorname{#1}^{#3} \left( #3 \right)}
            {\operatorname{#1}      \left( #4 \right)}
        }
}

\DeclareDocumentCommand{\funcil}{m o o m}
{%
    \IfValueTF{#2}
        {%
        \IfValueTF{#3}
            {\operatorname{#1}_{#2}^{#3} (#4)}
            {\operatorname{#1}_{#2}      (#4)}
        }
        {%
        \IfValueTF{#3}
            {\operatorname{#1}^{#3} (#3)}
            {\operatorname{#1}      (#4)}
        }
}

\DeclareDocumentCommand{\funcsb}{m o o m}
{%
    \IfValueTF{#2}
        {%
        \IfValueTF{#3}
            {\operatorname{#1}_{#2}^{#3} \left[ #4 \right]}
            {\operatorname{#1}_{#2}      \left[ #4 \right]}
        }
        {%
        \IfValueTF{#3}
            {\operatorname{#1}^{#3} \left[ #3 \right]}
            {\operatorname{#1}      \left[ #4 \right]}
        }
}

\DeclareDocumentCommand{\vecf}{m o o}
{%
    \IfValueTF{#2}
        {%
        \IfValueTF{#3}
            {\bm{#1}_{#2}^{#3}}
            {\bm{#1}_{#2}     }
        }
        {%
        \IfValueTF{#3}
            {\bm{#1}^{#3}}
            {\bm{#1}     }
        }
}

%%
% Book metadata
\title{\proposaltitle}
\author[Nitish Thatte]{Nitish Thatte}
%\publisher{Publisher of This Book}

% Generates the index
\usepackage{makeidx}
\makeindex

\begin{document}

% Front matter
%\frontmatter

% r.3 full title page
%\maketitle
\begin{titlepage}
	\begin{fullwidth}
	\centering
    \phantom.
    \vspace{0.5in}
    {\huge{\proposaltitle}\par}
    \vspace{0.5in}
    
    Nitish Thatte \\
    \today \\
    \vspace{0.9 in}
    
    The Robotics Institute \\
    Carnegie Mellon University \\
    Pittsburgh, PA 15213
    \vspace{0.9 in}
    
   	{\it Thesis Committee:}\\
    Hartmut Geyer (Chair)\\
    Chris Atkeson\\
    Aaron Johnson\\
    Steven Collins, Stanford University\\
    Elliott Rouse, University of Michigan\\
    \vspace{0.9 in}
   
   	{\it Submitted in partial fulfillment of the requirements\\ for the degree of Doctor of Philosophy}\\
    \vspace{0.9 in}
    
    Copyright \copyright \the\year \ Nitish Thatte.
 	\end{fullwidth}
\end{titlepage}

%\chapter*{Abstract}

Amputees face a number of gait deficits due to a lack of control and power from
their mechanically-passive prostheses. Of crucial importance among these
deficits are those related to balance, as falls and a fear of falling can cause
an avoidance of activity that leads to further debilitation. In this thesis, we
investigate the role that prosthesis control strategies play in maintaining
balance with a powered robotic transfemoral prosthesis. Our approach involves
comparing state-of-the-art prosthesis controllers on a common platform and
learning from this experiment to propose two new prosthesis control strategies
that directly address observed causes of falls in both the swing and stance
phases.

We begin by designing and manufacturing our own powered transfemoral prosthesis
capable of large torques for stumble recovery and accurate reproduction of
desired torques from different control strategies. We also propose a pair of
optimization methods that allow us to select prosthesis control parameters using
qualitative preference feedback from the user.

Next, we test a hypothesis that a stance control approach based on a model of
the human neuromuscular system may help improve gait robustness and user
satisfaction over the commonly used impedance control method. This hypothesis
stems from previous research applying neuromuscular control to simulated biped
models and to powered ankle prostheses that suggests that this approach can adapt
to changes in speed, incline, and rough ground. While our experiment did not
find a significant reduction in falls using neuromuscular control, it did reveal
that a lack of robust gait phase estimation caused a large number of falls for
the impedance control strategy and that both controllers suffered from trips
during swing.

Therefore, we next proposed and tested two new control strategies that directly
address these causes of falls. In the first, we use information from an inertial
measurement unit and LIDAR distance sensor to estimate the position, orientation
and future trajectory of the hip. This information is then used to plan
trajectories for the prosthesis' knee and ankle that avoid tripping during
swing. Second, we propose using an extended Kalman filter to improve phase
estimation during stance. We show the resulting control strategy significantly
reduced the number of falls compared impedance control when users step on uneven
terrain. These results demonstrate the importance of state estimation for
improving gait stability.


% r.5 contents
\tableofcontents

\listoffigures

\listoftables

% r.7 dedication
%\cleardoublepage
%~\vfill
%\begin{doublespace}
%\noindent\fontsize{18}{22}\selectfont\itshape
%\nohyphenation
%Dedicated to those who appreciate \LaTeX{} 
%and the work of \mbox{Edward R.~Tufte} 
%and \mbox{Donald E.~Knuth}.
%\end{doublespace}
%\vfill
%\vfill

%%
% Start the main matter (normal chapters)
\mainmatter

\section{Introduction}

To date, there have been many proposed controllers for prostheses, which we
reviewed in \cref{sec:back_walking_control}. However, there has been a dearth of
studies directly comparing the merits and detriments of these different
strategies. 

In this work, we provide one such study by comparing the neuromuscular (NM) and
impedance (IMP) control strategies in a similar manner as in
\cref{sec:control_sim}.  We seek to make this comparison as objective as
possible. To do this, we minimized the experimenter's influence on controller
parameter selection by using the dueling bandits parameter selection method
presented in \cref{sec:preference_optimization}. This method comprises of two
parts: 1) We first generated parameters for both controllers through offline
optimizations that try to match the controller output to able-bodied gait data
of different subjects. 2) We used a preference-based optimization that allows
users to select their preferred parameter set. Finally, we also replaced the
hand tuning of offset angles that we used in \cref{sec:preference_optimization}
(\cref{eq:bias_param}) with an iterative learning procedure.

We had also hoped to compare neuromuscular and impedance control to the
continuous phase-based control proposed by \citet{quintero2016preliminary}.
However, we were unable to achieve a consistent walking pattern with this
control strategy. We present our results trying to implement this control
strategy later in this thesis in \cref{sec:polar_phase}.

We are primarily interested in potential robustness improvements provided by
neuromuscular prosthesis control, as predicted by the simulation results
presented in \cref{sec:control_sim}. To examine controller robustness, we tested
both controllers with able-bodied subjects walking with their preferred
parameters at constant speed and with treadmill velocity disturbances. We then
evaluated the user ratings, number of falls, reasons for falls, and gait
variability for each condition. 

%\section{Background}\label{sec:intro_background} 

\begin{enumerate}
\item Prosthesis Design and low-level control
\item Mid level Prosthesis Controllers
    \begin{enumerate}
        \item Time Based/Echo Control
        \item Impedance 
            \begin{enumerate}
                \item lawson2014 used in this paper
                \item other versions with nonlinear ankle torques/4 states
                \item speed adaptation with impedance and quasistiffness
                \item slopes
                \item dempster shafer rules for more robust transitions
            \end{enumerate}
        \item Neuromuscular Ankle
        \item Phase Variable
            \item using COP
            \item using hip angle/integral
            \item splitting up into phases again
        \item Aaron Ames - Nonlienar control 
        \item CLME
        \item Minjerk Swing
    \end{enumerate}
\item High Level Control
    \begin{enumerate}
        \item Mode detection stairs/slopes/etc - Classifiers, EMG, LIDAR
        \item Trip Detection and recovery strategies - Zhang, Shirtoa, Goldfarb
        \item Faliure modes of high level prosthesis control - Zhang
    \end{enumerate}
\end{enumerate}

%\input{neuromuscular_model}
%\input{completed_work}
%\input{completed_tuning}
%\input{proposed_work}

% The back matter contains appendices, bibliographies, indices, glossaries, etc.

\backmatter

\bibliography{references}
\bibliographystyle{plainnat}

\end{document}
